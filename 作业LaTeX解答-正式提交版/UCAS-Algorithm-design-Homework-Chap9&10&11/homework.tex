\documentclass{article}

%
% 引入模板的style文件
%
\usepackage{homework}

\setCJKmainfont{SimSun}[AutoFakeBold] %宋体加粗
\setCJKsansfont{SimHei}[AutoFakeBold] %黑体加粗


\usepackage{minted} %配合minted宏包进行好看的高亮
\usepackage{currfile} %配合minted宏包进行好看的高亮
\usepackage{caption} %配合minted宏包进行好看的高亮
\usepackage{tcolorbox} %配合minted宏包进行好看的高亮
\usepackage{xcolor} %配合minted宏包进行好看的高亮
\tcbuselibrary{skins} %配合minted宏包进行好看的高亮
\tcbuselibrary{minted} %配合minted宏包进行好看的高亮
\usemintedstyle{paraiso-dark} %配合minted宏包进行好看的高亮
\usepackage{framed} 
\usepackage{amsmath}


%
% 封面
%


\title{
	\includegraphics[width=0.6\textwidth]{images/title/ucas_logo 1.pdf}\\
    \vspace{1in}
    \textmd{\textbf{\hmwkClass}}\\
	\textmd{\Large{\textbf{\hmwkClassID}}}\\
    \textmd{\textbf{\hmwkTitle}}\\
    \normalsize\vspace{0.1in}\large{\hmwkCompleteTime }\\
    \vspace{0.1in}\large{\textit{\hmwkClassInstructor\ }}\\
    \vspace{1in}
	\includegraphics[width=0.25\textwidth]{images/title/Cyber.jpg}\\
	\vspace{1in}
}

\author{
	\hmwkAuthorName \\ 
	\hmwkAuthorStuID \\
	\hmwkAuthorInst \\
	\hmwkAuthorzhuanye \\
	\hmwkAuthorfangxiang
	}
\date{}

\renewcommand{\part}[1]{\textbf{\large Part \Alph{partCounter}}\stepcounter{partCounter}\\}


%
% 正文部分
%
\begin{document}


\maketitle


%\include{chapters/ch01}
%\include{chapters/ch02}
%\include{chapters/ch03}
%\include{chapters/ch04}
%\include{chapters/ch05}


\pagebreak


\begin{homeworkProblem}
	设集合$S=\left\{ x_1,x_2,\cdots ,x_n \right\}$, $x_i$出现的概率为$0<p_i<1,\displaystyle \sum_{i=1}^n{p_i}=1$. 试设计一个算法, 按照$S$的概率分布对$S$的元素进行随机选择(一个).

	\solution 可以设计出以下的随机算法, 代码有C++版本、也有python版本(对应的输出结果紧跟代码之后):
\begin{tcblisting}{listing engine=minted,boxrule=0.1mm,
colback=blue!5!white,colframe=blue!75!black,
listing only,left=5mm,enhanced,sharp corners=all,
overlay={\begin{tcbclipinterior}\fill[red!20!blue!20!white] (frame.south west)
rectangle ([xshift=5mm]frame.north west);\end{tcbclipinterior}},
minted language=c++,
minted style=tango,
minted options={fontsize=\small,breaklines,autogobble,linenos,numbersep=3mm}}
#include <bits/stdc++.h> //万能头文件
using namespace std;

int random_pick(vector<pair<int, double>> nums_probs) {
    int res;
    random_device rd; //定义一个非确定性的真随机生成器
    uniform_real_distribution<double> U(0, 1); //随机数分布对象
    double x = U(rd); //在[0,1)内随机选取实数作为x
    double cumu_prob = 0.0; //初始化累积概率
    for(auto num_prob : nums_probs) {
        cumu_prob += num_prob.second;
        if(x < cumu_prob) {
            res = num_prob.first;
            break;
        }
    }
    return res;
}

int main() {
    vector<pair<int, double>> nums_probs = {
        {1, 0.2}, {2, 0.1}, {3, 0.6}, {4, 0.1}
    };
    cout << "自写算法的抽取结果: " << random_pick(nums_probs) << endl;
    random_device rd;
    discrete_distribution<> d({0.2, 0.1, 0.6, 0.1});
    cout << "调用C++11中算法的抽取结果: " << d(rd) + 1 << endl;
}
\end{tcblisting}
上述C++代码对应的输出结果为:
\begin{tcblisting}{listing engine=minted,boxrule=0.1mm,
colback=blue!5!white,colframe=blue!75!black,
listing only,left=5mm,enhanced,sharp corners=all,
overlay={\begin{tcbclipinterior}\fill[red!20!blue!20!white] (frame.south west)
rectangle ([xshift=5mm]frame.north west);\end{tcbclipinterior}},
minted language=python,
minted style=tango,
minted options={fontsize=\small,breaklines,autogobble,linenos,numbersep=3mm}}
开始运行...
自写算法的抽取结果: 3
调用C++11中算法的抽取结果: 3
运行结束.
\end{tcblisting}

\begin{tcblisting}{listing engine=minted,boxrule=0.1mm,
colback=blue!5!white,colframe=blue!75!black,
listing only,left=5mm,enhanced,sharp corners=all,
overlay={\begin{tcbclipinterior}\fill[red!20!blue!20!white] (frame.south west)
rectangle ([xshift=5mm]frame.north west);\end{tcbclipinterior}},
minted language=python,
minted style=tango,
minted options={fontsize=\small,breaklines,autogobble,linenos,numbersep=3mm}}
import random
import numpy as np
 
def random_pick(some_list, probabilities):
    x = random.uniform(0, 1)
    cumulative_probability = 0.0
    for item, item_probability in zip(some_list, probabilities):
        cumulative_probability += item_probability
        if x < cumulative_probability: break
    return item

if __name__ == '__main__':
    some_list = [1, 2, 3, 4]
    probabilities = [0.2, 0.1, 0.6, 0.1]
    print("自写算法的抽取结果: ", random_pick(some_list, probabilities))
    print("调用numpy中算法的抽样结果: ", np.random.choice(some_list, 1, probabilities))
\end{tcblisting}
上述python代码对应的输出结果为:
\begin{tcblisting}{listing engine=minted,boxrule=0.1mm,
colback=blue!5!white,colframe=blue!75!black,
listing only,left=5mm,enhanced,sharp corners=all,
overlay={\begin{tcbclipinterior}\fill[red!20!blue!20!white] (frame.south west)
rectangle ([xshift=5mm]frame.north west);\end{tcbclipinterior}},
minted language=python,
minted style=tango,
minted options={fontsize=\small,breaklines,autogobble,linenos,numbersep=3mm}}
开始运行...
自写算法的抽取结果: 3
调用numpy中算法的抽样结果: [3]
运行结束.
\end{tcblisting}
	显然该随机算法的复杂度主要取决于for循环, 因此上述算法的时间复杂度为$O(n)$, 其中$n$为集合$S$中的元素个数. 而空间复杂度显然为$O(1)$.

\end{homeworkProblem}

\begin{homeworkProblem}
	设计概率算法, 求解$365!/\left(340!\cdot 365^{25}\right)$.

	\solution 该问题的数是概率论中著名的生日问题的解答. 在$k$个人中, 至少2个人有相同生日的概率为$$1-\frac{365!}{\left( 365-k \right) !\cdot 365^k}
	$$
	根据斯特林公式以及近似公式:
	$$
	n!\sim \sqrt{2\pi n}\left( \frac{n}{e} \right) ^n(n\to \infty),\quad \text{ln} \left( 1+x \right) \sim x-\frac{x^2}{2}(x\to 0)
	$$
	因此有:
	\begin{align}
		\frac{n!}{\left( n-k \right) !\cdot n^k}&\sim \frac{\sqrt{2\pi n}}{\sqrt{2\pi \left( n-k \right)}}\cdot \frac{\left( \frac{n}{e} \right) ^n}{\left( \frac{n-k}{e} \right) ^{n-k}\cdot n^k} \notag
		\\
		&\sim \left( \frac{n}{n-k} \right) ^{n-k}\cdot e^{-k}=\text{exp} \left[ \left( n-k \right) \ln \left( 1+\frac{k}{n-k} \right) -k \right] \notag
		\\
		&\sim \text{exp} \left\{ \left( n-k \right) \left[ \frac{k}{n-k}-\frac{\frac{k^2}{\left( n-k \right) ^2}}{2} \right] -k \right\} \notag
		\\
		&\sim \text{exp} \left\{ -\frac{k^2}{2\left( n-k \right)} \right\} \sim \text{exp} \left\{ -\frac{k^2}{2n} \right\} \notag
	\end{align}
	因此可得
	$$
	\frac{n!}{\left( n-k \right) !\cdot n^k}\bigg|_{n=365}^{k=25}=\frac{365!}{340!\cdot 365^{25}}\approx \text{exp} \left\{ -\frac{25^2}{730} \right\} =0.424788
	$$
\end{homeworkProblem}

\begin{homeworkProblem}
	设计一个Las Vegas随机算法, 求解电路板布线问题. 将该算法与分支限界算法结合, 观察求解效率.

	\solution 该算法的设计核心是: \textbf{采用随机放置位置策略并结合分支限界算法}. 算法的C++代码如下所示: 
\begin{tcblisting}{listing engine=minted,boxrule=0.1mm,
colback=blue!5!white,colframe=blue!75!black,
listing only,left=5mm,enhanced,sharp corners=all,
overlay={\begin{tcbclipinterior}\fill[red!20!blue!20!white] (frame.south west)
rectangle ([xshift=5mm]frame.north west);\end{tcbclipinterior}},
minted language=c++,
minted style=tango,
minted options={fontsize=\small,breaklines,autogobble,linenos,numbersep=3mm}}
#include <bits/stdc++.h> //万能头文件, 刷题时可以用, 大型项目千万不能用
using namespace std;

//表示方格位置上的结构体
struct position{
    int row;
    int col;
};

//分支限界算法
bool FindPath(
    position start, position finish, int &PathLen,
    position *&path, int **grid, int m, int n
) { //找到最短布线路径, 若找得到则返回true, 否则返回false
    //起点终点重合则不用布线
    if((start.row == finish.row) && (start.col == finish.col)) {
        PathLen = 0;
        return true;
    }

    //设置方向移动坐标值: 东南西北
    position offset[4];
    offset[0].row = 0;
    offset[0].col = 1; //右移
    offset[1].row = 1;
    offset[1].col = 0; //下移
    offset[2].row = 0;
    offset[2].col = -1; //左移
    offset[3].row = -1;
    offset[3].col = 0; //上移

    int NumNeighBlo = 4; //相邻的方格数
    position here, nbr;
    here.row = start.row; //设置当前方格, 即搜索单位
    here.col = start.col;
    //由于0和1用于表示方格的开放和封闭, 故距离: 2-0 3-1
    grid[start.row][start.col] = 0; //-2表示强, -1表示可行, -3表示不能当作路线
    //队列式搜索, 标记可达相邻方格
    queue<position> q_FindPath;
\end{tcblisting}
\begin{tcblisting}{listing engine=minted,boxrule=0.1mm,
colback=blue!5!white,colframe=blue!75!black,
listing only,left=5mm,enhanced,sharp corners=all,
overlay={\begin{tcbclipinterior}\fill[red!20!blue!20!white] (frame.south west)
rectangle ([xshift=5mm]frame.north west);\end{tcbclipinterior}},
minted language=c++,
minted style=tango,
minted options={fontsize=\small,breaklines,autogobble,linenos,numbersep=3mm}}
    do {
        int num = 0; //方格未标记个数
        position selectPosition[5]; //保存选择位置
        for(int i = 0; i < NumNeighBlo; i++) {
            //到达四个方向
            nbr.row = here.row + offset[i].row;
            nbr.col = here.col + offset[i].col;
            if(grid[nbr.row][nbr.col] == -1) { //该方格未标记
                grid[nbr.row][nbr.col] = grid[here.row][here.col] + 1;
                if((nbr.row == finish.row) && (nbr.col == finish.col)) {
                    break;
                }
                selectPosition[num].row = nbr.row;
                selectPosition[num].col = nbr.col;
                num++;
            }
        }
        if(num > 0) { //如果标记, 则在这么多个未标记个数中随机选择一个位置(本算法核心)
            q_FindPath.push(selectPosition[rand()%(num)]); //随机选一个入队
        }
        if((nbr.row == finish.row) && (nbr.col == finish.col)) {
            break; //是否到达目标位置finish
        }
        //判断活结点队列是否为空
        if(q_FindPath.empty() == true) return false; // 无解
        //访问队首元素出队
        here = q_FindPath.front();
        q_FindPath.pop();
    } while (true);

    //构造最短布线路径
    PathLen = grid[finish.row][finish.col];
    path = new position[PathLen]; //路径
    //从目标位置finish开始向起始位置回溯
    here = finish;
    for(int j = PathLen - 1; j >= 0; j--) {
        path[j] = here;
        //找前驱位置
        for(int i = 0; i <= NumNeighBlo; i++) {
            nbr.row = here.row + offset[i].row;
            nbr.col = here.col + offset[i].col;
            if(grid[nbr.row][nbr.col] == j) {//距离+2正好是前驱位置
                break;
            }
        }
        here = nbr;
    }
    return true;
}
\end{tcblisting}
\begin{tcblisting}{listing engine=minted,boxrule=0.1mm,
colback=blue!5!white,colframe=blue!75!black,
listing only,left=5mm,enhanced,sharp corners=all,
overlay={\begin{tcbclipinterior}\fill[red!20!blue!20!white] (frame.south west)
rectangle ([xshift=5mm]frame.north west);\end{tcbclipinterior}},
minted language=c++,
minted style=tango,
minted options={fontsize=\small,breaklines,autogobble,linenos,numbersep=3mm}}
int main() {
    cout << "---------分支限界算法之布线问题---------" << endl;
    int path_len, path_len1, m, n;
    position *path, *path1, start, finish, start1, finish1;
    cout << "在一个m*n的棋盘上, 请分别输入m和n, 代表行数和列数, 然后输入回车" << endl;
    cin >> m >> n;
    //创建棋盘格
    int **grid = new int * [m + 2], **grid1 = new int * [m + 2];
    for(int i = 0; i < m + 2; i++) {
        grid[i] = new int[n + 2];
        grid1[i] = new int[n + 2];
    }
    //初始化棋盘
    for(int i = 1; i <= m; i++) {
        for(int j = 1; j <= n; j++) {
            grid[i][j] = -1;
        }
    }
    //设置方格阵列的围墙
    for(int i = 0; i <= n + 1; i++) {
        grid[0][i] = grid[m + 1][i] = -2; //上下的围墙
    }
    for(int i = 0; i <= m + 1; i++) {
        grid[i][0] = grid[i][n + 1] = -2; //左右的围墙
    }
    cout << "初始化棋盘格和加围墙" << endl;
    cout << "------------------" << endl;
    for(int i = 0; i < m + 2; i++) {
        for(int j = 0; j < n + 2; j++) {
            cout << grid[i][j] << " ";
        }
        cout << endl;
    }
    cout << "------------------" << endl;
    cout << "请输入已经占据的位置(行坐标, 列坐标), 代表此位置不能布线" << endl;
    cout << "例如输入2 2(然后输入回车), 表示坐标(2,2)不能布线;" <<
    "当输入坐标为0 0(然后输入回车)表示结束输入" << endl;
    //添加已经布线的棋盘格
    while(true) {
        int ci, cj;
        cin >> ci >> cj;
        if(ci > m || cj > n) {
            cout << "输入非法!";
            cout << "行坐标<" << m << ", 列坐标<" << n << "当输入的坐标为0,0时结束输入" << endl;
            continue;
        } else if (ci == 0 || cj == 0) {
            break;
        } else {
            grid[ci][cj] = -3;
        }
    }
\end{tcblisting}
\begin{tcblisting}{listing engine=minted,boxrule=0.1mm,
colback=blue!5!white,colframe=blue!75!black,
listing only,left=5mm,enhanced,sharp corners=all,
overlay={\begin{tcbclipinterior}\fill[red!20!blue!20!white] (frame.south west)
rectangle ([xshift=5mm]frame.north west);\end{tcbclipinterior}},
minted language=c++,
minted style=tango,
minted options={fontsize=\small,breaklines,autogobble,linenos,numbersep=3mm}}
    //布线前的棋盘格
    cout << "布线前的棋盘格" << endl;
    cout << "------------------" << endl;
    for(int i = 0; i < m + 2; i++) {
        for(int j = 0; j < n + 2; j++) {
            cout << grid[i][j] << " ";
        }
        cout << endl;
    }
    cout << "------------------" << endl;
    cout << "请输入起点位置坐标" << endl;
    cin >> start.row >> start.col;
    cout << "请输入终点位置坐标" << endl;
    cin >> finish.row >> finish.col;
    clock_t starttime, endtime;
    starttime = clock(); //程序开始时间
    srand((unsigned) time (NULL)); //初始化时间种子, 是随机选择的关键
    int time = 0; //为假设运行次数
    start1 = start, finish1 = finish, path_len1 = path_len, path1 = NULL; //初始值拷贝
    for(int i = 0; i < m + 2; i++) {
        for(int j = 0; j < n + 2; j++) {
            grid1[i][j] = grid[i][j];
        }
    }
    bool result = FindPath(start1, finish1, path_len1, path1, grid1, m, n);
    while(result == 0 && time < 1000) { //尝试次数最多为1000次
        //初始值拷贝
        start1 = start, finish1 = finish, path_len1 = path_len, path1 = NULL;
        for(int i = 0; i < m + 2; i++) {
            for(int j = 0; j < n + 2; j++) {
                grid1[i][j] = grid[i][j];
            }
        }
        time++;
        cout << endl;
        cout << "没有找到路线, 第" << time << "次尝试" << endl;
        result = FindPath(start1, finish1, path_len1, path1, grid1, m, n);
    }
    endtime = clock(); //程序结束时间

    if(result == 1) {
        cout << "------------------" << endl;
        cout << "$ 代表围墙" << endl;
        cout << "# 代表已经占据的点" << endl;
        cout << "* 代表布线路线" << endl;
        cout << "= 代表还没有布线的点" << endl;
        cout << "------------------" << endl;
        for(int i = 0; i <= m + 1; i++) {
            for(int j = 0; j <= n + 1; j++) {
                if(grid1[i][j] == -2) cout << "$ ";
                else if(grid1[i][j] == -3) cout << "# ";
\end{tcblisting}

\begin{tcblisting}{listing engine=minted,boxrule=0.1mm,
colback=blue!5!white,colframe=blue!75!black,
listing only,left=5mm,enhanced,sharp corners=all,
overlay={\begin{tcbclipinterior}\fill[red!20!blue!20!white] (frame.south west)
rectangle ([xshift=5mm]frame.north west);\end{tcbclipinterior}},
minted language=c++,
minted style=tango,
minted options={fontsize=\small,breaklines,autogobble,linenos,numbersep=3mm}}
                else {
                    int r;
                    for(r = 0; r < path_len1; r++) {
                        if(i == path1[r].row && j == path1[r].col) {
                            cout << "* ";
                            break;
                        }
                        if(i == start1.row && j == start1.col) {
                            cout << "* ";
                            break;
                        }
                    }
                    if(r == path_len1) cout << "= ";
                }
            }
            cout << endl;
        }
        cout << "------------------" << endl;
        cout << "路径坐标和长度" << endl;
        cout << endl;
        cout << "(" << start1.row << "," << start1.col << ")" << " ";
        for(int i = 0; i < path_len1; i++) {
            cout << "(" << path1[i].row << "," << path1[i].col << ")" << " ";
        }
        cout << endl;
        cout << endl;
        cout << "路径长度: " << path_len1 + 1 << endl;
        cout << endl;
        time++;
        cout << "布线完毕, 共查找" << time << "次" << endl;
        cout << "算法运行时间为: " << (endtime - starttime) << "ms" << endl;
    } else {
        cout << endl;
        cout << "经过多次尝试, 依然没有找到路线" << endl;
    }
    return 0;
}
\end{tcblisting}
	上述代码的关键之处在于:
	\begin{itemize}
		\item P5页的第13行代码, 这里是当前点相邻四个点是否可以放置, 如果可以放置用selectPostion保存下来, 并用num记录有多少个位置可以放置;
		\item P5页的第19行代码, 这里利用上面保存的可以放置的点, 然后\textbf{随机选取其中一个加入队列}, 这就是Las Vegas算法的精髓;
		\item P7页的第17行代码, 作用是初始化时间种子, 是伪随机生成器的关键, 即是随机选择的关键.
	\end{itemize}
	\textbf{结果分析:}
	\begin{itemize}
		\item 测试样例1: $3\times 3$棋盘, 代码的交互输出过程如下:
\begin{tcblisting}{listing engine=minted,boxrule=0.1mm,
colback=blue!5!white,colframe=blue!75!black,
listing only,left=5mm,enhanced,sharp corners=all,
overlay={\begin{tcbclipinterior}\fill[red!20!blue!20!white] (frame.south west)
rectangle ([xshift=5mm]frame.north west);\end{tcbclipinterior}},
minted language=c++,
minted style=tango,
minted options={fontsize=\small,breaklines,autogobble,linenos,numbersep=3mm}}
开始运行...
---------分支限界算法之布线问题---------
在一个m*n的棋盘上, 请分别输入m和n, 代表行数和列数, 然后输入回车
3 3
初始化棋盘格和加围墙
------------------
-2 -2 -2 -2 -2 
-2 -1 -1 -1 -2 
-2 -1 -1 -1 -2 
-2 -1 -1 -1 -2 
-2 -2 -2 -2 -2 
------------------
请输入已经占据的位置(行坐标, 列坐标), 代表此位置不能布线
例如输入2 2(然后输入回车), 表示坐标(2,2)不能布线;当输入坐标为0 0(然后输入回车)表示结束输入
2 1
2 3
3 3
0 0
布线前的棋盘格
------------------
-2 -2 -2 -2 -2 
-2 -1 -1 -1 -2 
-2 -3 -1 -3 -2 
-2 -1 -1 -3 -2 
-2 -2 -2 -2 -2 
------------------
请输入起点位置坐标
1 1
请输入终点位置坐标
3 1

没有找到路线, 第1次尝试

没有找到路线, 第2次尝试
------------------
$ 代表围墙
# 代表已经占据的点
* 代表布线路线
= 代表还没有布线的点
------------------
$ $ $ $ $ 
$ * * = $ 
$ # * # $ 
$ * * # $ 
$ $ $ $ $ 
------------------
路径坐标和长度
(1,1) (1,2) (2,2) (3,2) (3,1) 
路径长度: 5
布线完毕, 共查找3次
算法运行时间为: 39ms
运行结束.
\end{tcblisting}
	\item 测试样例2: $5\times 5$棋盘, 代码的交互输出过程如下:
\begin{tcblisting}{listing engine=minted,boxrule=0.1mm,
colback=blue!5!white,colframe=blue!75!black,
listing only,left=5mm,enhanced,sharp corners=all,
overlay={\begin{tcbclipinterior}\fill[red!20!blue!20!white] (frame.south west)
rectangle ([xshift=5mm]frame.north west);\end{tcbclipinterior}},
minted language=c++,
minted style=tango,
minted options={fontsize=\small,breaklines,autogobble,linenos,numbersep=3mm}}
---------分支限界算法之布线问题---------
在一个m*n的棋盘上, 请分别输入m和n, 代表行数和列数, 然后输入回车
5 5
请输入已经占据的位置(行坐标, 列坐标), 代表此位置不能布线
例如输入2 2(然后输入回车), 表示坐标(2,2)不能布线;当输入坐标为0 0(然后输入回车)表示结束输入
3 1
3 2
3 4
3 5
4 5
0 0
布线前的棋盘格
------------------
-2 -2 -2 -2 -2 -2 -2 
-2 -1 -1 -1 -1 -1 -2 
-2 -1 -1 -1 -1 -1 -2 
-2 -3 -3 -1 -3 -3 -2 
-2 -1 -1 -1 -1 -3 -2 
-2 -1 -1 -1 -1 -1 -2 
-2 -2 -2 -2 -2 -2 -2 
------------------
请输入起点位置坐标
1 1
请输入终点位置坐标
5 2
没有找到路线, 第1次尝试
没有找到路线, 第2次尝试
没有找到路线, 第3次尝试
------------------
$ 代表围墙
# 代表已经占据的点
* 代表布线路线
= 代表还没有布线的点
------------------
$ $ $ $ $ $ $ 
$ * = = = = $ 
$ * * * = = $ 
$ # # * # # $ 
$ = = * = # $ 
$ = * * = = $ 
$ $ $ $ $ $ $ 
------------------
路径坐标和长度
(1,1) (2,1) (2,2) (2,3) (3,3) (4,3) (5,3) (5,2) 
路径长度: 8
布线完毕, 共查找4次
算法运行时间为: 61ms
\end{tcblisting}
	\item 测试样例2: $10\times 10$棋盘, 代码的交互输出过程如下:
\begin{tcblisting}{listing engine=minted,boxrule=0.1mm,
colback=blue!5!white,colframe=blue!75!black,
listing only,left=5mm,enhanced,sharp corners=all,
overlay={\begin{tcbclipinterior}\fill[red!20!blue!20!white] (frame.south west)
rectangle ([xshift=5mm]frame.north west);\end{tcbclipinterior}},
minted language=c++,
minted style=tango,
minted options={fontsize=\small,breaklines,autogobble,linenos,numbersep=3mm}}
---------分支限界算法之布线问题---------
在一个m*n的棋盘上, 请分别输入m和n, 代表行数和列数, 然后输入回车
10 10
布线前的棋盘格
------------------
-2 -2 -2 -2 -2 -2 -2 -2 -2 -2 -2 -2 
-2 -1 -1 -1 -1 -1 -1 -1 -1 -1 -1 -2 
-2 -1 -1 -1 -1 -1 -1 -1 -1 -1 -1 -2 
-2 -1 -1 -1 -1 -1 -1 -1 -1 -1 -1 -2 
-2 -1 -1 -1 -1 -1 -1 -1 -1 -1 -1 -2 
-2 -3 -3 -3 -3 -1 -1 -3 -3 -3 -1 -2 
-2 -1 -1 -1 -1 -1 -1 -1 -1 -1 -1 -2 
-2 -1 -1 -1 -1 -1 -1 -1 -1 -1 -1 -2 
-2 -1 -1 -1 -1 -1 -1 -1 -1 -1 -1 -2 
-2 -1 -1 -1 -1 -1 -1 -1 -1 -1 -1 -2 
-2 -1 -1 -1 -1 -1 -1 -1 -1 -1 -1 -2 
-2 -2 -2 -2 -2 -2 -2 -2 -2 -2 -2 -2 
------------------
请输入起点位置坐标
1 1
请输入终点位置坐标
9 9

没有找到路线, 第1次尝试
没有找到路线, 第2次尝试
没有找到路线, 第3次尝试
没有找到路线, 第4次尝试
------------------
$ 代表围墙
# 代表已经占据的点
* 代表布线路线
= 代表还没有布线的点
------------------
$ $ $ $ $ $ $ $ $ $ $ $ 
$ * = = = = = = = = = $ 
$ * * * = = = = = = = $ 
$ = = * * = = = = = = $ 
$ = = = * * = = = = = $ 
$ # # # # * = # # # = $ 
$ = = = = * = = = = = $ 
$ = = = = * * * = = = $ 
$ = = = = = = * = = = $ 
$ = = = = = = * * * = $ 
$ = = = = = = = = = = $ 
$ $ $ $ $ $ $ $ $ $ $ $ 
------------------
路径坐标和长度
(1,1) (2,1) (2,2) (2,3) (3,3) (3,4) (4,4) (4,5) (5,5) (6,5) (7,5) (7,6) (7,7) (8,7) (9,7) (9,8) (9,9) 
路径长度: 17
布线完毕, 共查找5次
算法运行时间为: 73ms
\end{tcblisting}
	\end{itemize}
	由此可见, 结合随机化和分支限界的Las Vegas算法的求解效率还是相当不错的.
\end{homeworkProblem}


\begin{homeworkProblem}
	\textbf{判断正误:}
	\begin{itemize}
		\item Las Vegas算法不会得到不正确的解. (\quad )
		\item Monte Carlo算法不会得到不正确的解. (\quad )
		\item Las Vegas算法总能求得一个解. (\quad)
		\item Monte Carlo算法总能求得一个解. (\quad)
	\end{itemize}
	\solution 
	\begin{itemize}
		\item 正确, 拉斯维加斯算法不会得到不正确的解. 一旦用拉斯维加斯算法找到一个解, 这个解就一定是正确解. 但有时用拉斯维加斯算法找不到解.
		\item 错误, Monte Carlo算法每次都能得到问题的解, 但不保证所得解的正确性. 请注意, 可以在Monte Carlo算法给出的解上加一个验证算法, 如果正确就得到解, 如果错误就不能生成问题的解, 这样Monte Carlo算法便转化为了Las Vegas算法.
		\item 错误, Las Vegas算法并不能保证每次都能得到一个解, 但是如果一旦某一次得到解, 那么就一定是正确的.
		\item 正确, Monte Carlo算法每次运行都能给出一个解, 但正确性就不能保证了.
	\end{itemize}
\end{homeworkProblem}


\begin{homeworkProblem}
	设Las Vegas算法获得解的概率为$p(x)\geq \delta,0<\delta <1$, 则调用$k$次算法后, 获得解的概率为:_____.

	\solution 不妨求一下调用$k$次算法后, 求解失败(即$k$次调用都求解失败)的概率: 
	$$
	P\left( \text{失败} \right) =\left( 1-p\left( x \right) \right) ^k\le \left( 1-\delta \right) ^k\Rightarrow P\left( \text{成功} \right) =1-P\left( \text{失败} \right) \ge 1-\left( 1-\delta \right) ^k
	$$
	即获得解的概率至少为$1-\left( 1-\delta \right) ^k\to 1(\text{当}k\to \infty)$.
\end{homeworkProblem}

\begin{homeworkProblem}
	对于判定问题$\Pi$的Monte Carlo算法, 当返回false(true)时解总是正确的, 但当返回true(false)时解可能有错误, 该算法是______.
	\begin{align}
		&\left( \text{A} \right) . \text{偏真的}\textit{Monte}\,\,\textit{Carlo}\text{算法}\quad \quad \quad \quad \quad \quad \quad \quad \quad \quad \quad \quad\left( \text{B} \right) . \text{偏假的}\textit{Monte}\,\,\textit{Carlo}\text{算法} \notag
		\\
		&\left( \text{C} \right) . \text{一致的}\textit{Monte}\,\,\textit{Carlo}\text{算法}\quad \quad \quad \quad \quad \quad \quad \quad \quad \quad \quad \quad\left( \text{D} \right) . \text{不一致的}\textit{Monte}\,\,\textit{Carlo}\text{算法} \notag
	\end{align}
	\solution 答案选B, 只要将偏真的Monte Carlo算法的定义中的true/false互换即可得到偏真的Monte Carlo算法的定义.
\end{homeworkProblem}

\pagebreak

\begin{homeworkProblem}
	\textbf{判断正误:}
	\begin{itemize}
		\item 一般情况下, 无法有效判定Las Vegas算法所得解是否正确. (\quad)
		\item 一般情况下, 无法有效判定Monte Carlo算法所得解是否正确. (\quad)
		\item 虽然在某些步骤引入随机选择, 但Sherwood算法总能求得问题的一个解, 且所求得的解总是正确的. (\quad)
		\item 虽然在某些步骤引入随机选择, 但Sherwood算法总能求得问题的一个解, 但一般情况下, 无法有效判定所求得的解是否正确. (\quad)
	\end{itemize}
	\solution
	\begin{itemize}
		\item 错误, Las Vegas算法并不能保证每次都能得到解, 但是如果一旦某一次得到解, 那么就一定是正确的.
		\item 错误, 虽然Monte Carlo算法每次运行都能给出一个解, 可能是错的也可能是对的, 但是可以通过检验解来有效判定其正确性. 判定解的正确性跟算法本身没有多大关系, 只要代进去验证即可. 特殊点在于, 只要Las Vegas算法求得解了, 那么就一定是正确的, 就不用再浪费时间来判定了; 但是对于Monte Carlo算法的所得解, 必须要进行正确性检验.
		\item 正确, Sherwood算法总能求得问题的一个解, 且所求得的解总是正确的.
		\item 错误.
	\end{itemize}
\end{homeworkProblem}

\pagebreak


\begin{homeworkProblem}
    \textbf{装箱问题:} 任给$n$件物品, 物品$j$的重量为$w_j, 1\leq j\leq n$, 限制每只箱子装入物品的总重量不超过$B$, 其中$B$和$w_j$都是正整数, 且$w_j\leq B,1\leq j\leq n$. 要求用最少的箱子装入所有物品, 怎么装法?

    考虑下述近似算法-\textbf{首次适合算法(FF)}: 按照输入顺序装物品, 对每一件物品, 依次检查每一只箱子, 只要能装得下就把它装入, 只有在所有已经打开的箱子都装不下这件物品时, 才打开一个新箱子. 证明: \textbf{FF算法}是2-近似的, 即任给实例$I$, 都有$$\textbf{FF}(I)< 2\text{OPT}(I)$$

    \solution 当$\textbf{FF}(I)=1$时, 显然$\textbf{FF}(I)=\text{OPT}(I)=1$. 如果$\textbf{FF}(I)>1$, 记$\displaystyle W=\sum_{i=1}^n{w_i}$. 因为任何两只箱子的重量之和大于$B$. 
    \begin{itemize}
        \item 因此当$\textbf{FF}(I)$为偶数时, $$\displaystyle W>\frac{B}{2}\textbf{FF}(I)$$
        \item 当$\textbf{FF}(I)$为奇数时, 设最重的箱子重量为$B_1$, 则有$$\displaystyle W>\frac{B}{2}\left( \textbf{FF}\left( I \right) -1 \right) +B_1>\frac{B}{2}\textbf{FF}\left( I \right)$$
    \end{itemize}
    故总有$\displaystyle W>\frac{B}{2}\textbf{FF}(I)$, 即$\displaystyle \textbf{FF}(I)<\frac{2W}{B}$. 而显然$\displaystyle \text{OPT}(I)\geq \frac{W}{B}$, 因此证得$\textbf{FF}(I)<2\text{OPT}(I),\Box$.
\end{homeworkProblem}


\begin{homeworkProblem}
    设无向图$G=\left< V,E \right>, V_1\cup V_2=V, V_1\cap V_2=\emptyset$, 称$$\left( V_1,V_2 \right) =\left\{ \left( u,v \right) \big|\left( u,v \right) \in E,\text{且}u\in V_1,v\in V_2 \right\} 
    $$
    为$G$的割集, $(V_1,V_2)$中的边称为割边, 不在$(V_1,V_2)$中的边称作非割边.

    \textbf{最大割集问题:} 任给无向图$G=\left<V,E\right>$, 求$G$的边数最多的割集. 考虑下述求最大割集问题的\textbf{局部改进算法(MCUT)}: 令$V_1=V,V_2=\emptyset$. 如果存在顶点$u$, 在$u$关联的边中非割边多于割边, 如果$u\in V_1$, 则把$u$移到$V_2$中; 如果$u\in V_2$, 则把$u$移到$V_1$中, 直到不存在这样的顶点为止, 取此时得到的$(V_1,V_2)$作为解.

    证明: MCUT是\textbf{2-近似算法}, 即对任一实例$I$, 都有$$\text{OPT}(I)\leq 2\textbf{MCUT}(I)$$

    \solution 根据算法, 每一个顶点关联的割边数大于等于关联的非割边数. 对所有的顶点求和, 每条边出现2次, 故所有的割边数大于等于所有非割边数. 从而$\displaystyle \textbf{MCUT}(I)\geq \frac{|E|}{2}$. 又显然有$\text{OPT}(I)\leq |E|$, 于是证得$\text{OPT}(I)\leq 2\textbf{MCUT}(I),\Box.$\footnote{最小割集问题是多项式时间可解的, 而最大割集问题是$\mathcal{NPH}$的.}
    \newpage
\end{homeworkProblem}


\pagebreak

\begin{homeworkProblem}
    \textbf{双机调度问题(优化形式)}: 有2台相同的机器和$n$项作业$J_1,J_2,\cdots,J_n$, 每一项作业可以在任一台机器上处理, 没有顺序的限制, 作业$J_i$的处理时间为正整数$t_i,1\leq i \leq n$. 要求把$n$项作业分配给这2台机器使得完成时间最短, 即把$\left\{ 1,2,\cdots ,n \right\} $划分为$I_1$和$I_2$, 使得$$\text{max} \left\{ \sum_{i\in I_1}{t_i},\sum_{i\in I_2}{t_i} \right\} 
    $$
    最小.

    令$\displaystyle D=\left\lfloor \frac{1}{2}\sum_{i=1}^n{t_i} \right\rfloor ,B\left( i \right) =\left\{ t \bigg|t=\sum_{j\in S}{t_j}\le D,S\subseteq \left\{ 1,2,\cdots ,i \right\} \right\} ,0\le i\le n$. $B(i)$包括所有前$i$项作业中任意项(可以是0项)作业的处理时间之和, 只要这个和不超过所有作业处理时间之和的二分之一. 试给出关于$B(i)$的递推公式, 并利用这个递推公式设计双机调度问题的伪多项式时间算法, 进而设计这个问题的完全多项式时间近似算法.

    \solution 递推公式如下所示:$$B\left( 0 \right) =\left\{ 0 \right\} ,B\left( i \right) =B\left( i-1 \right) \cup \left\{ t\big|t-t_i\in B\left( i-1 \right) ,t_i\le t\le D \right\} ,i=1,2,\cdots ,n
    $$
    于是显然有
    $$
    \text{OPT}\left( I \right) =\sum_{i=1}^n{t_i}-\text{max} B\left( n \right) 
    $$
    于是我们可以给出如下的\hyperlink{alg:DP}{\textbf{DP}}算法:
    \begin{algorithm}[H]
		\begin{algorithmic}[1]
		\Require{$n$个作业的处理时间$t_1,t_2,\cdots,t_n$}
		\Ensure{处理好的作业集$J$}
        \State 令$\displaystyle D\gets\left\lfloor \frac{1}{2}\sum_{i=1}^n{t_i} \right\rfloor, B(0)\gets0$;
		\For{$i=1$; $i \leq n$; $i$++}
            \State 令$B(i)\gets B(i-1)$;
            \For{$t= t_i$; $t\leq D$; $t$++}
                \If{$t-t_i\in B(i-1)$}
                    \State $B(i)\gets B(i)\cup \{t\}$;
                \EndIf
            \EndFor
		\EndFor
        \State 令$t\gets \text{max}B(n),J\gets\emptyset,i\gets n$;
        \For{$i= n$; $i\geq 1$; $i$-\,-}
            \If{$t-t_i\in B(i-1)$}
                \State $J\gets J\cup \{i\},t\gets t-t_i$;
            \EndIf
            \If{$t\leq 0$}
                \State 输出$J$, \textbf{break};
            \EndIf
        \EndFor
		\State \textbf{end \{DP\}};
		\end{algorithmic}
		\caption{\textbf{DP}算法}
		\label{alg:DP}
	\end{algorithm}
    DP的时间复杂度为$O(nD)=O\left( n^2t_{\text{max}} \right)$, 其中$t_{\text{max}}=\text{max}\left\{ t_i\big|i=1,2,\cdots ,n \right\}$. 这是伪多项式时间算法. 进而设计出下述的完全多项式时间近似算法-\hyperlink{alg:FPTAS}{\textbf{FPTAS}}:
    \begin{algorithm}[H]
		\begin{algorithmic}[1]
		\Require{$n$个作业的处理时间$t_1,t_2,\cdots,t_n$和$\epsilon > 0$}
		\Ensure{处理好的作业集$J$}
        \State 令$t_{\text{max}}\gets\text{max} \left\{ t_i|i=1,2,\cdots ,n \right\}$;
        \State 令$\displaystyle b\gets\text{max} \left\{ \left\lfloor t_{\text{max}}\bigg/\left( 1+\frac{1}{\epsilon} \right) n \right\rfloor ,1 \right\}$;
        \State 令$t_i'\gets t_i/b,i=1,2,\cdots,n$;
        \State 以$t_i'(i=1,2,\cdots,n)$为输入并运行DP算法;
        \State 输出处理好的作业集$J$;
		\State \textbf{end \{FPTAS\}};
		\end{algorithmic}
		\caption{\textbf{FPTAS}算法}
		\label{alg:FPTAS}
	\end{algorithm}
    \textbf{FPTAS}的时间复杂度为$\displaystyle O\left(n^2t_{\text{max}}'\right)=O\left( n^2 t_{\text{max}}/b\right)=O\left( n^3\left(1+\frac{1}{\epsilon}\right) \right)$.

    当$b=1$, \textbf{FPTAS}得到最优解. 不妨设$b>1$, $b\left( t_i'-1 \right) <t_i\le bt_i'$. 对任意的$S\subseteq \left\{ 1,2,\cdots ,n \right\} $, 有
    \begin{align}
        & b\sum_{i\in S}{t_i'}-b\left| S \right|<\sum_{i\in S}{t_i}\le b\sum_{i\in S}{t_i'}, \notag
        \\
        & 0\le b\sum_{i\in S}{t_i'}-\sum_{i\in S}{t_i}<b\left| S \right|\le bn \tag{$\ast$} \label{eq:*}
    \end{align}
    记最优解为$J^{\ast}$, \textbf{FPTAS}得到的近似解为$J$和$\displaystyle J'=\left\{ 1,2,\cdots,n \right\}\backslash J,\text{OPT}(I)=\sum_{i\in J^{\ast}}{t_i}, \textbf{FPTAS}(I)=\sum_{i\in J'}{t_i}$, 于是有
    \begin{align}
        \textbf{FTPAS}\left( I \right) -\text{OPT}\left( I \right) &=\sum_{i\in J'}{t_i}-\sum_{i\in J^{\ast}}{t_i} \notag
        \\
        &=\left( \sum_{i\in J'}{t_i}-b\sum_{i\in J'}{t_i'} \right) +\left( b\sum_{i\in J'}{t_i'}-b\sum_{i\in J^{\ast}}{t_i'} \right) +\left( b\sum_{i\in J^{\ast}}{t_i'}-\sum_{i\in J^{\ast}}{t_i} \right) \notag
    \end{align}
    根据\eqref{eq:*}式可知上述第一项$\leq 0$. $J'$是关于$\left\{ t_i' \right\}$的最优解, 第二项也$\leq 0$, 于是得到$$\textbf{FTPAS}\left( I \right) -\text{OPT}\left( I \right) \le b\sum_{i\in J^{\ast}}{t_i'}-\sum_{i\in J^{\ast}}{t_i}<bn\le t_{\text{max}}\bigg/\left( 1+\frac{1}{\epsilon} \right) \le \textbf{FTPAS}\left( I \right) \bigg/\left( 1+\frac{1}{\epsilon} \right) 
    $$
    化简得到$\textbf{FTPAS}(I)\leq (1+\epsilon)\text{OPT}(I)$, 因此\textbf{FTPAS}是双机调度问题的完全多项式时间近似算法,$\Box.$
\end{homeworkProblem}

\pagebreak

\begin{homeworkProblem}
    \textbf{顶点覆盖问题:} 任给一个图$G=\left<V,E\right>$, 求$G$的顶点数最少的顶点覆盖. 复习顶点覆盖问题的近似算法及其证明.

    \solution \,\,\hyperlink{alg:MVC}{\textbf{MVC}}算法如下所示: 
    \begin{algorithm}[H]
		\begin{algorithmic}[1]
		\Require{图$G=\left<V,E\right>$}
		\Ensure{最小顶点覆盖$V'$}
        \State $V'\gets \emptyset$, $e_1\gets E$;
        \While{$e_1\neq \emptyset$}
            \State 从$e_1$中任选一条边$(u,v)$;
            \State $V'\gets V' \cup \{u,v\}$;
            \State 从$e_1$中删去与$u$和$v$相关联的所有边;
        \EndWhile
        \State \Return $V'$;
		\State \textbf{end \{MVC\}};
		\end{algorithmic}
		\caption{算法\textbf{MVC}$(G)$}
		\label{alg:MVC}
	\end{algorithm}
    显然算法MVC的时间复杂度为$O(m),m=|E|$. 记$\left|V'\right|=2k$, $V'$中的顶点是$k$条边的端点, 这$k$条边互不关联. 为了覆盖这$k$条边则需要$k$个顶点, 从而$\text{OPT}(I)\geq k$. 于是有$$\frac{\text{MVC}\left( I \right)}{\text{OPT}\left( I \right)}\le \frac{2k}{k}=2
    $$
    故MVC是最小顶点覆盖问题的2-近似算法,$\Box.$
\end{homeworkProblem}

\begin{homeworkProblem}
	\textbf{判断正误:}
	\begin{itemize}
		\item 旅行商问题存在多项式时间近似方案. (\quad)
		\item 0/1背包问题存在多项式时间近似方案. (\quad)
		\item 0/1背包问题的贪心算法(单位价值高优先装入)是绝对近似算法. (\quad)
		\item 多机调度问题的贪心近似算法(按输入顺序将作业分配给当前最小负载机器)是$\epsilon$-近似算法. (\quad)
	\end{itemize}
	\solution
	\begin{itemize}
		\item 错误. 根据教材可知, 旅行商问题不存在多项式时间近似算法, 除非$\mathcal{P}=\mathcal{NP}$. 如果存在的话, 那么就可以证得$\mathcal{P}=\mathcal{NP}$, 即可以拿图灵奖了.
		\item 正确, \textbf{PTAS算法}就是0/1背包问题的多项式时间近似方案.
		\item 错误, 0/1背包问题的贪心算法\textbf{不是}绝对近似算法.
		\item 正确, 多机调度问题的贪心近似算法有GMPS和DGMPS分别是2-近似和3/2-近似算法.
	\end{itemize}
\end{homeworkProblem}

\pagebreak



\begin{homeworkProblem}
    设旅行商问题的解表示为$D=F=\left\{ S|S=\left( i_1,i_2,\cdots ,i_n \right) ,i_1,i_2,\cdots ,i_n\text{是}1,2,\cdots ,n\text{的一个排列} \right\}$, 邻域定义为2-OPT(即$S$中的两个元素对换), 求$S=(3,1,2,4)$的邻域$N(S)$.

    \solution 将$S$中的两个元素对换即可得到$N(S)$:
    $$
    N\left( S \right) =\left\{ \left( 1,3,2,4 \right) ,\left( 2,1,3,4 \right) ,\left( 4,1,2,3 \right) ,\left( 3,2,1,4 \right) ,\left( 3,4,2,1 \right) ,\left( 3,1,4,2 \right) \right\} 
    $$
\end{homeworkProblem}

\begin{homeworkProblem}
    0/1背包问题的解记作$X=(x_1,x_2,\cdots,x_n),x_i\in \{0,1\},i=1,2,\cdots,n$. 邻域定义为$$\displaystyle N\left( X \right) =\left\{ Y \Bigg |\sum_{i=1}^n{\left| y_i-x_i \right|}\le 1 \right\},X=(1,1,0,0,1)$$
    求邻域$N(X)$.

    \solution 每次只允许一个分量变化即可求出邻域$N(X)$:
    $$
    N\left( X \right) =\left\{ \left( 0,1,0,0,1 \right) ,\left( 1,0,0,0,1 \right) ,\left( 1,1,1,0,1 \right) ,\left( 1,1,0,1,1 \right) ,\left( 1,1,0,0,0 \right) \right\} 
    $$
\end{homeworkProblem}

\begin{homeworkProblem}
    写出禁忌搜索算法的主要步骤.

    \solution 禁忌搜索算法的主要步骤如下算法\ref{alg:jinji}中所示:
    \begin{algorithm}[H]
		\begin{algorithmic}[1]
        \State 选定一个初始可行解$\boldsymbol{x}^{\text{cb}}$并初始化禁忌表$H\gets \{\}$;
        \While{不满足停止规则}
            \State 在$\boldsymbol{x}^{\text{cb}}$的邻域中选出满足禁忌要求的候选集$\text{Can-}N\left( \boldsymbol{x}^{\text{cb}} \right)$;
            \State 从该候选集中选出一个评价值最佳的解$\boldsymbol{x}^{\text{lb}}$;
            \State 令$\boldsymbol{x}^{\text{cb}}\gets \boldsymbol{x}^{\text{lb}}$并更新记录$H$;
        \EndWhile
		\end{algorithmic}
		\caption{禁忌搜索算法步骤}
		\label{alg:jinji}
	\end{algorithm}
\end{homeworkProblem}

\begin{homeworkProblem}
    禁忌对象特赦可以基于影响力规则: 即特赦影响力大的禁忌对象. 影响力大什么含义? 举例说明该规则的好处.

    \solution 影响力大意味着有些对象变化对目标值影响很大. 如0/1背包问题, 当包中无法装入新物品时, 特赦体积大的分量来避开局部最优解.
\end{homeworkProblem}

\pagebreak


\begin{homeworkProblem}
    \textbf{判断正误:}
    \begin{itemize}
        \item 禁忌搜索中, 禁忌某些对象是为了避免领域中的不可行解. (\quad )
        \item 禁忌长度越大越好. (\quad )
        \item 禁忌长度越小越好. (\quad )
    \end{itemize}
    \solution 
    \begin{itemize}
        \item 错误, 选取禁忌对象是为了引起解的变化, 根本目的在于避开邻域内的局部最优解而不是不可行解.
        \item 错误, 禁忌长度短了则可能陷入局部最优解.
        \item 错误, 禁忌长度长了则导致计算时间长.
    \end{itemize}
\end{homeworkProblem}

\begin{homeworkProblem}
    写出模拟退火算法的主要步骤.

    \solution 模拟退火算法的主要步骤如下算法\ref{alg:moni}中所示:
    \begin{algorithm}[H]
		\begin{algorithmic}[1]
        \State 任选初始解$x_0$并初始化$x_i\gets x_0,k\gets 0, t_0\gets t_{\text{max}}$(初始温度);
        \While{$k\leq k_{\text{max}}$ \&\& $t_k\geq T_f$}
            \State 从邻域$N(x_i)$中随机选择$x_j$, 即$x_j\gets _RN(x_i)$;
            \State 计算$\Delta f_{ij}=f(x_j)-f(x_i)$;
            \If{$\Delta f_{ij}\leq 0$ || $\text{exp}\left( -\Delta f_{ij}/t_k \right)>\text{RANDOM}(0,1)$}
                \State $x_i\gets x_j$;
            \EndIf
            \State $t_{k+1}\gets d(t_k)$;
            \State $ k\gets k+1$;
        \EndWhile
		\end{algorithmic}
		\caption{模拟退火算法步骤}
		\label{alg:moni}
	\end{algorithm}
\end{homeworkProblem}



\begin{homeworkProblem}
    为避免陷入局部最优(小), 模拟退火算法以概率$\text{exp}\left( -\Delta f_{ij}/t_k \right)$接受一个退步(比当前最优解差)的解, 以跳出局部最优. 试说明参数$t_k,\Delta f_{ij}$对是否接受退步解的影响.

    \solution 很明显, 当$t_k$较大时, 接受退步解的概率越大; 当$\Delta f_{ij}$较大时, 接受退步解的概率越小.
\end{homeworkProblem}


\begin{homeworkProblem}
    下面属于模拟退火算法实现的关键技术问题的有______.
    \begin{align}
		\left( \text{A} \right) . \text{初始温度}\quad \quad \quad \left( \text{B} \right) . \text{温度下降控制} \quad \quad \quad 
		\left( \text{C} \right) . \text{邻域定义}\quad \quad \quad \left( \text{D} \right) . \text{目标函数} \notag
	\end{align}

    \solution 模拟退火算法实现的关键技术问题有\textbf{邻域的定义(构造)、起始温度的选择、温度下降方法、每一温度的迭代长度以及算法终止规则}. 因此选择(A), (B), (C).
\end{homeworkProblem}

\begin{homeworkProblem}
    用遗传算法解某些问题, $\text{fitnees}=f(x)$可能导致适应函数难以区分这些染色体. 请给出一种解决办法.

    \solution 采用非线性加速适应函数如下
    $$
    \text{fitness}\left( x \right) =\left\{ \begin{matrix}
        \displaystyle \frac{1}{f_{\text{max}}-f\left( x \right)},&		f\left( x \right) <f_{\text{max}}\\
        M>0,&		f\left( x \right) =f_{\text{max}}\\
    \end{matrix} \right. 
    $$
    其中$M$是一个充分大的值, $f_{\text{max}}$是当前最优值. 
\end{homeworkProblem}

\begin{homeworkProblem}
    用非常规编码染色体实现的遗传算法, 如TSP问题使用$1,2,\cdots,n$的排列编码, 简单交配会产生什么问题? 如何解决?

    \solution 后代可能会出现非可行解, 因此需要通过罚值和交叉新规则来解决.
\end{homeworkProblem}

\begin{homeworkProblem}
    下面属于遗传算法实现的关键技术问题的有______.
    \begin{align}
		\left( \text{A} \right) . \text{解的编码}\quad \quad \quad \left( \text{B} \right) . \text{初始种群的选择} \quad \quad \quad 
		\left( \text{C} \right) . \text{邻域定义}\quad \quad \quad \left( \text{D} \right) . \text{适应函数} \notag
	\end{align}

    \solution 遗传算法实现的关键技术问题有\textbf{解的编码、适应函数、初始种群的选取、交叉规则以及终止规则}. 因此选择(A), (B), (D).
\end{homeworkProblem}

至此, 9-10-11章的所有练习都已做完. 

\vspace{3cm}

\begin{figure}[H]  % 这里记得用[H]
    \centering
    \includegraphics[width=0.7\linewidth]{images/title/ucas_logo 1.pdf}
    %\caption{ucas-logo}
    \label{fig:ucas-logo}
\end{figure}

% 引用文献
\bibliographystyle{unsrt}  % unsrt:根据引用顺序编号
\bibliography{refs}


\end{document}
