\documentclass{article}

%
% 引入模板的style文件
%
\usepackage{homework}

\setCJKmainfont{SimSun}[AutoFakeBold] %宋体加粗
\setCJKsansfont{SimHei}[AutoFakeBold] %黑体加粗


\usepackage{minted} %配合minted宏包进行好看的高亮
\usepackage{currfile} %配合minted宏包进行好看的高亮
\usepackage{caption} %配合minted宏包进行好看的高亮
\usepackage{tcolorbox} %配合minted宏包进行好看的高亮
\usepackage{xcolor} %配合minted宏包进行好看的高亮
\tcbuselibrary{skins} %配合minted宏包进行好看的高亮
\tcbuselibrary{minted} %配合minted宏包进行好看的高亮
\usemintedstyle{paraiso-dark} %配合minted宏包进行好看的高亮



%
% 封面
%


\title{
	\includegraphics[width=0.6\textwidth]{images/title/ucas_logo 1.pdf}\\
    \vspace{1in}
    \textmd{\textbf{\hmwkClass}}\\
	\textmd{\Large{\textbf{\hmwkClassID}}}\\
    \textmd{\textbf{\hmwkTitle}}\\
    \normalsize\vspace{0.1in}\large{\hmwkCompleteTime }\\
    \vspace{0.1in}\large{\textit{\hmwkClassInstructor\ }}\\
    \vspace{1in}
	\includegraphics[width=0.25\textwidth]{images/title/Cyber.jpg}\\
	\vspace{1in}
}

\author{
	\hmwkAuthorName \\ 
	\hmwkAuthorStuID \\
	\hmwkAuthorInst \\
	\hmwkAuthorzhuanye \\
	\hmwkAuthorfangxiang
	}
\date{}

\renewcommand{\part}[1]{\textbf{\large Part \Alph{partCounter}}\stepcounter{partCounter}\\}


%
% 正文部分
%
\begin{document}


\maketitle


%\include{chapters/ch01}
%\include{chapters/ch02}
%\include{chapters/ch03}
%\include{chapters/ch04}
%\include{chapters/ch05}


\pagebreak

\begin{homeworkProblem}
	试确定下述程序的关键操作数, 该函数实现一个$m\times n$矩阵与一个$n\times p$矩阵之间的乘法:
	
\begin{tcblisting}{listing engine=minted,boxrule=0.1mm,
	colback=blue!5!white,colframe=blue!75!black,
	listing only,left=5mm,enhanced,sharp corners=all,
	overlay={\begin{tcbclipinterior}\fill[red!20!blue!20!white] (frame.south west)
	rectangle ([xshift=5mm]frame.north west);\end{tcbclipinterior}},
	minted language=c++,
	minted style=tango,
	minted options={fontsize=\normalsize,breaklines,autogobble,linenos,numbersep=3mm}}
    template <class T>
    void Mult(T **a, T **b, int m, int n, int p) {
        //m×n矩阵A与n×p矩阵B相成得到m×p矩阵C
        for(int i = 0; i < m; i++) {
            for(int j = 0; j < p ;j++) {
                T sum = 0;
                Tfor(int k = 0; k < n; k++)
                Tsum += a[i][k]*b[k][j];
                C[i][j] = sum;
            }
        }
    }
\end{tcblisting}
	
	
	\solution
	\\

	此算法的关键操作为\textbf{第8行}, 该语句的操作数为2 (1次加法和1次乘法). 则三层循环的总关键操作数为:$$T=\sum_{i=0}^{m-1}{\sum_{j=0}^{p-1}{\sum_{k=0}^{n-1}{2}}}=2mnp
	$$
	故该算法的时间复杂度为$$T\left( m,n,p \right) =O\left( 2mnp \right) =O\left( mnp \right) $$
	
\end{homeworkProblem}


\begin{homeworkProblem}
	函数MinMax用来查找数组$a[0:n-1]$中的最大元素和最小元素, 以下给出两个程序. 令$n$为实例特征. 试问: 在各个程序中, $a$中元素之间的比较次数在最坏情况下各是多少?

\begin{tcblisting}{listing engine=minted,boxrule=0.1mm,
	colback=blue!5!white,colframe=blue!75!black,
	listing only,left=5mm,enhanced,sharp corners=all,
	overlay={\begin{tcbclipinterior}\fill[red!20!blue!20!white] (frame.south west)
	rectangle ([xshift=5mm]frame.north west);\end{tcbclipinterior}},
	minted language=c++,
	minted style=tango,
	minted options={fontsize=\normalsize,breaklines,autogobble,linenos,numbersep=3mm}}
    /*找最大最小元素 (方法一)*/
    template <class T>
    bool MinMax(T a[], int n, int& Min, int& Max) {
        //寻找a[0:n-1]中的最小元素与最大元素
        //如果数组中的元素数目小于1, 则返回false
        if(n<1) return false;
        Min=Max=0; //初始化
        for(int i=1; i<n; i++) {
            if(a[Min]>a[i]) Min=i;
            if(a[Max]<a[i]) Max=i;
        }
        return true;
    }
\end{tcblisting}

\begin{tcblisting}{listing engine=minted,boxrule=0.1mm,
	colback=blue!5!white,colframe=blue!75!black,
	listing only,left=5mm,enhanced,sharp corners=all,
	overlay={\begin{tcbclipinterior}\fill[red!20!blue!20!white] (frame.south west)
	rectangle ([xshift=5mm]frame.north west);\end{tcbclipinterior}},
	minted language=c++,
	minted style=tango,
	minted options={fontsize=\normalsize,breaklines,autogobble,linenos,numbersep=3mm}}
    /*找最大最小元素 (方法二)*/
    template <class T>
    bool MinMax(T a[], int n, int& Min, int& Max) {
        //寻找a[0:n-1]中的最小元素与最大元素
        //如果数组中的元素数目小于1, 则返回false
        if(n<1) return false;
        Min=Max=0; //初始化
        for(int i=1; i<n; i++) {
            if(a[Min]>a[i]) Min=i;
            else if(a[Max]<a[i]) Max=i;
        }
        return true;
    }
\end{tcblisting}
	
	
	\solution
	\\

	不论数组$a$是单调递增还是单调递减, for循环内部的两次判断都得执行, 所以方法1在任何情况下的元素比较次数都为$2\times (n-1)$; 而对于方法2来说, 当数组$a$单调递减(最好情况)时, for循环内部的第一个判断条件一定满足, 那么else if这个判断自然就不用执行了, 即此情形下的元素比较次数为$1\times (n-1)$. 但当数组$a$单调递增(最坏情况)时, 第一个循环条件在各轮循环中都不能满足, 所以紧跟着需要执行后边的else if判断, 即此情形下的元素比较次数为$2\times (n-1)$.
\end{homeworkProblem}



\begin{homeworkProblem}
	证明以下关系式不成立: (1). $10n^2+9=O\left( n \right) $; \quad (2). $n^2\log n=\Theta \left( n^2 \right)$.
	\\

	\solution
	\\

	\begin{proof}

		\textbf{(1).} 考虑极限$\displaystyle \lim_{n\rightarrow \infty} \frac{10n^2+9}{n}=+\infty >c\,\,\left( \forall c>0 \right) $, 故根据大$O$比率定理可知该式不成立;
		\\ 

		\textbf{(2).} 考虑极限$\displaystyle \lim_{n\rightarrow \infty} \frac{n^2\log n}{n^2}=+\infty >c\,\,\left( \forall c>0 \right) $, 故根据$\Theta$比率定理可知该式不成立.
	\end{proof}
	
\end{homeworkProblem}

\begin{homeworkProblem}
	按照渐近阶从低到高的顺序排列以下表达式:
	$$4 n^2, \log n, 3^n, 20 n, n^{2 / 3}, n !$$

	\solution
	\\

	根据Stirling公式:$$n!\sim \sqrt{2\pi n}\left( \frac{n}{e} \right) ^n (\text{as}\, n\to +\infty)$$

	可知有渐进阶的顺序为$$\log n\ll n^{2 / 3} \ll 20n=\Theta \left( n \right) \ll 4n^2=\Theta \left( n^2 \right) \ll 3^n\ll n!
	$$
\end{homeworkProblem}

\begin{homeworkProblem}

	(1). 假设某算法在输入规模是$n$时为$T(n)=3\ast 2^n$. 在某台计算机上实现并完成该算法的时间是$t$秒.现有另一台计算机,其运行速度为第一台的64倍. 那么,在这台计算机上用同一算法在$t$秒内能解决规模为多大的问题?

	(2). 若上述算法改进后的新算法的时间复杂度为$T(n)=n^2$, 则在新机器上用$t$秒时间能解决输入规模为多大的问题?

	(3). 若进一步改进算法, 最新的算法的时间复杂度为$T(n)=8$, 其余条件不变, 在新机器上运行, 在$t$秒内能够解决输入规模为多大的问题?
	\\

	\solution
	\\

	\textbf{(1).} 设问题规模为$M$, 则新机器上的求解时间为$t=3\times 2^M/64$, 老机器的求解时间$t= 3\times 2^n$,即解得$M=n+6$;

	\textbf{(2).} 同理设问题规模为$M$, 则新机器上的求解时间为$t=M^2/64=n^2$, 老机器的求解时间$t=n^2$,即解得$M=8n$;

	\textbf{(3).} 因为该算法的时间复杂度是常数阶的, 也就意味着问题规模不影响求解时间(当问题规模很大时), 所以在任何(可以运行该算法的)机器上, $t$秒内可以解决任意规模的问题.

\end{homeworkProblem}


\begin{homeworkProblem}
	考虑下述选择排序算法\ref{alg:选择排序}所示: 
	
	\begin{algorithm}[H]
		\begin{algorithmic}[1]
		\Require{$n$个不等整数的数组$A[1..n]$}
		\Ensure{按递增次序排序的$A$}
		\For{$i:=1$ to $n$}
			\For{$ j:=i+1 $ to $n$}
				\If{$A[j]<A[i]$}
				\State $A[i] \leftrightarrow  A[j]$
				\EndIf
			\EndFor
		\EndFor
		\State 输出排序后的数组$A$
		\end{algorithmic}
		\caption{选择排序}
		\label{alg:选择排序}
	\end{algorithm}
	问: (1)最坏情况下做多少次比较运算?

	(2)最坏情况下做多少次交换运算?在什么输入时发生?
	\\

	\solution
	\\

	\textbf{(1).} 任意情况下要比较$\left( n-1 \right) +\left( n-2 \right) +\cdots +1=\frac{1}{2}n\left( n-1 \right) $次. 再者, 此处不存在所谓的最好或最坏情况, 不论数组$A$是逆序还是升序排列, 计算机不可能因为提前得知$A$的所有情况而不执行判断语句, 即每次循环都需要进行比较运算;

	\textbf{(1).} 最坏情况: 输入数组内元素为降序排列. 此时做$\left( n-1 \right) +\left( n-2 \right) +\cdots +1=\frac{1}{2}n\left( n-1 \right) $次交换运算. 当数组$A$内元素为升序排列时, 则交换次数为0 (即为最好情况).
\end{homeworkProblem}

\begin{homeworkProblem}
	考虑下面的每对函数$f(n)$和$g(n)$, 比较他们的阶.

	(1). $f\left( n \right) =\frac{1}{2}\left( n^2-n \right) ,\,\, g\left( n \right) =6n$; \quad \quad (2). $f\left( n \right) =n+2\sqrt{n},\,\, g\left( n \right) =n^2$;

	(3). $f\left( n \right) =n+n\log n,\,\, g\left( n \right) =n\sqrt{n}$; \quad (4). $f\left( n \right) =\log \left( n! \right) ,\,\, g\left( n \right) =n^{1.05}$;
	\\

	\solution
	\\

	\textbf{(1)}. $f\left( n \right) =\Theta \left( n^2 \right) \gg \Theta \left( n \right) =g\left( n \right)$; \quad \quad \quad \,\,\,\,\textbf{(2)}. $f\left( n \right) =\Theta \left( n \right) \ll \Theta \left( n^2 \right) =g\left( n \right) $;

	\textbf{(3)}. $f\left( n \right) =\Theta \left( n\log n \right) \ll \Theta \left( n^{1.5} \right) =g\left( n \right) $; \quad \textbf{(4)}. 根据\textbf{斯特林公式}可知:$$\log \left( n! \right) \sim \log \left( \sqrt{2\pi n}\left( \frac{n}{e} \right) ^n \right) =\frac{1}{2}\log \left( 2\pi n \right) +n\left( \log n-\log e \right) \sim n\log n \, \left( \mathrm{as} \,n\rightarrow \infty \right) 
	$$

	所以有$f\left( n \right) =\Theta \left( n\log n \right) \ll \Theta \left( n^{1.05} \right) =g\left( n \right) $.
\end{homeworkProblem}


\begin{homeworkProblem}
	在表\ref{原始表格}中填入true或false.
	\begin{table}[H]
		\centering
		\begin{tabular}{|c|c|c|c|c|c|}
		\hline
		& $f(n)$       & $g(n)$            & $f(n)=O(g(n))$ & $f(n)=\Omega(g(n))$ & $f(n)=\Theta(g(n))$ \\ \hline
		1 & $2n^3+3n$    & $100n^2+2n+100$   &                &                     &                     \\ \hline
		2 & $50n+\log n$ & $10n+\log \log n$ &                &                     &                     \\ \hline
		3 & $50n\log n$  & $10n \log \log n$ &                &                     &                     \\ \hline
		4 & $\log n$     & $\log^2 n$        &                &                     &                     \\ \hline
		5 & $n!$         & $5^n$             &                &                     &                     \\ \hline
		\end{tabular}
		\caption{原始表格}
		\label{原始表格}
	\end{table}

	\solution
	\\
	
	解答如下表\ref{解答}所示:
	\begin{table}[H]
		\centering
		\begin{tabular}{|c|c|c|c|c|c|}
		\hline
		& $f(n)$       & $g(n)$            & $f(n)=O(g(n))$ & $f(n)=\Omega(g(n))$ & $f(n)=\Theta(g(n))$ \\ \hline
		1 & $2n^3+3n$    & $100n^2+2n+100$   & False          & True                & False               \\ \hline
		2 & $50n+\log n$ & $10n+\log \log n$ & True           & True                & True                \\ \hline
		3 & $50n\log n$  & $10n \log \log n$ & False          & True                & False               \\ \hline
		4 & $\log n$     & $\log^2 n$        & True           & False               & False               \\ \hline
		5 & $n!$         & $5^n$             & False          & True                & False               \\ \hline
		\end{tabular}
		\caption{解答}
		\label{解答}
	\end{table}
\end{homeworkProblem}



\pagebreak

\begin{homeworkProblem}
	用迭代法求解下列递推方程: 

	(1). $\begin{cases}
		T\left( n \right) =T\left( n-1 \right) +n-1\\
		T\left( 1 \right) =0\\
	\end{cases}$; \quad (2). $\left\{ \begin{array}{l}
		T\left( n \right) =2T\left( \frac{n}{2} \right) +n-1\\
		T(1)=0\\
	\end{array},n=2^k \right.$
	\\

	\solution
	\\

	\textbf{(1)}. 易知$$\left\{ \begin{array}{c}
		T\left( 2 \right) -T\left( 1 \right) =1\\
		T\left( 3 \right) -T\left( 2 \right) =2\\
		\vdots\\
		T\left( n \right) -T\left( n-1 \right) =n-1\\
	\end{array} \right. \xrightarrow{\text{各式相加}}T\left( n \right) =1+2+\cdots +n-1=\frac{1}{2}n\left( n-1 \right) = \Theta \left(n^2\right)$$

	\textbf{(2)}. 令$n=2^k$,则易知有如下:
	\begin{align}
		T\left( 2^k \right) &=2T\left( 2^{k-1} \right) +1\cdot 2^k-1 \notag
		\\
		&=2^2T\left( 2^{k-2} \right) +2\cdot 2^k-\left( 1+2 \right) \notag
		\\
		&=2^3T\left( 2^{k-3} \right) +3\cdot 2^k-\left( 1+2+2^2 \right) \notag
		\\
		&\,\,\, \vdots \notag
		\\
		&=2^kT\left( 1 \right) +k\cdot 2^k-\left( 1+2+\cdots +2^{k-1} \right) \notag
		\\
		&=k\cdot 2^k+\left( 1-2^k \right) =\left( k-1 \right) \cdot 2^k+1 \notag
		\\
		&=n\log n-n+1=T(n)=\Theta\left( n\log n \right) \notag
	\end{align}
\end{homeworkProblem}

至此, Chap 1-2的作业解答完毕.


% 引用文献
\bibliographystyle{unsrt}  % unsrt:根据引用顺序编号
\bibliography{refs}


\end{document}
